\documentclass{ctexbook}
\setCJKmainfont{Noto Serif CJK SC}

\usepackage{graphicx}
\usepackage{xeCJKfntef}
\xeCJKsetup{underdot/symbol={\normalfont^^b7}}
\newcommand{\dotemph}[1]{\CJKunderdot{#1}}
\usepackage{xcolor}

\usepackage{lmodern}

% multi language %
%\usepackage{fontspec}
%\setmainfont{Noto Serif}
%\setsansfont{Noto Sans}

\usepackage{polyglossia}
%\setdefaultlanguage{chinese}
\setotherlanguages{russian}
\newfontfamily\russianfont[Script=Cyrillic]{Noto Serif}


% multicols %
\usepackage{multicol}

\title{\LARGE 古事记}
\author{普洱}
\date{二〇二三年八月二十日}


\begin{document}



\maketitle

\Large

\newpage
\textbf{目录}


\mbox


    说明················3


\mbox

    
    正文················4


\mbox

    
    跋·················45







\newpage
\textbf{\LARGE 说明}


\mbox


    文中所见旋转数学公式、蓝字意在模仿原稿痕迹,然囿于编者自身能力,未有一处齐备,姑作一笑。

    编者作文时,尚无安排之心,全为抒情而作。如今顾盼过往,不忍弃之于荒海,编此集以作了结。
    
    标 \# 文章,原稿无题。



\newpage

\textbf{\#}


\mbox


	今天太高兴了太高兴了太高兴了……!
	
	她,她从哪冒出来的?她怎么就忽地出现在我身后,怎么就朝我快乐地笑起来了?她又怎么一下子就溜了?这是精灵,这是我心中最棒的精灵!
	
	能认识这样一个人儿简直是人生中最最好,最最最可喜的事!她这可爱的人!\(^*\)


\mbox


    {\normalsize * 后来,她走了。}

\newpage

\textbf{\#}


\mbox


    有人活着,有人活在这里,那怎么样?他们存活,他们生活,他们在生活,他们活动,他们行动,他们不断地行动,他们交谈,他们进行不用思考的无意义的交谈,他们商榷,他们满足,他们愤怒,他们互相残杀,他们生育,他们进行毋须禁止但已被禁止的活动,他们辱骂,他们度过四月二十一日炎热的白日,他们忍受一月七日的冬夜,他们成功,他们不成功,他们背叛,他们制定法律而又践踏法律,他们离开,他们前往。

\newpage

\textbf{\#}


    我再一次回到了这窗边的位置,实际上我来到这里只有八月初那一次,然而这是属于我的,这是独属于我的位置。这是高三,我现在处于高三,我如今可以愈发笃定世界上有不止一个的我存在,但它们这些不同的我都如这窗边的位置般独属于同一个人,我在这世上不同场合伸出不同的神经,我感受到的,我目所及的,我所能领悟到的都随这场景所移,所变。在桃江,在\CJKunderdot{我的}桃江而非\CJKunderdot{别人的}桃江,我看到了几乎七年未见的雪,漫天飞舞,化作世界的泡沫,如同脑海中只浮现一次的童年记忆般。雪一直下,我看到大地上的白,它们一直下,直到寒冷不再,它们化为人们口中的言辞,消逝在那个无人察觉的夜。眼前的窗户覆着尘,也许至少十年没扫了吧,这老房子几乎再也不如从前那般承载着一家人的温馨,但,但是在那个冬天我们又重聚了,我们一家人又在这个漫漫的冬季于此重见,男人们打着牌,女人们取暖,窗外依然下着雪,我手脚被冻得几乎没有知觉,强逼着这些神经往头脑里走,化作为我心中永不消散的积雪,那些雪花白莹莹的,一点也不冷。望着天空的灰,我早已想到今日,我早已不敢设想将来,只想在那一刻化作时间的尘埃,随着这雪,随着我心中桃江的人们零散,桃江,桃江。

    八月前的一天,我第一个来到高三,我率先进入了这个\CJKunderdot{时刻},我是最先感知到的,手中拿着一封妈妈的信,我竟不敢,我竟坚持读完了,也许我很难向他们解释我已变化太多太多,但他们却毋需言辞便能证明爱我。那封信言意真切,鼓舞我向前,然而在随后的日子里,一切从前似乎都被空间的自我折磨与对话吞尽殆灭了,我忘记了所有,只留下了悲恸,我在学校,并非家中的我,我在此处没有什么行为可言,有时只是日复一日地不解、哭诉、谩骂以至于暴烈的怒火,一切过后我被自我折磨得只剩悲哀的皮囊。那不是\CJKunderdot{我}!那不是\CJKunderdot{我}!我在家里深切感受到变了,我变了,我畏怯了,我\CJKunderdot{疲倦}了,我只想向前跳出这地方,我要逃走!但回头看到他们,看到我爱的人,看到他们正支持着疲惫已极的我,我不忍,我不忍抛下为我写了那么多为了我不断下厨的妈妈,我不忍呵斥那岁月不再而为了我去五台山祈福的爸爸,我不忍抛下他们所有人,我看到我可爱的妹妹只愿她一切都好,我不忍见她的落泪。我那时正躺在床上,窗外是七八年来从未变化的天空,那儿的风吹进我的卧室,清新的风,谁下了床被妈妈听到?一点了,她竟还没有睡,她竟还听着我的声音,阳台之下的城,一点也没有变,树无非落了叶又再长叶,云无非落了雨又积雨,而他们,我的爸爸妈妈无非一日又一日地关心着我,我的记忆呢,无非忘却他们的爱而又苦苦索求。我的朋友们,庄如今比小时瘦,也更有担当了,杨与我互相鼓励,这些人都是爱我的,他们如我旁边的窗一样,都是我的,都是我这边的,没有借口再没有推辞之言让我逃了,如今只是前路,我的记忆,流吧,我必将我终将回报他们以我,以歌!

\newpage

\textbf{鱼的发光强度}


\mbox


    一天晚上,我正走在学校旁的那条乡间小路上,路两边是一排学生们种的椴木,有时有几只松鼠从树上爬下来,椴树上会有松鼠么?我不知道,或许是调皮的人带来放生的,有一个人坐在树底下脱他的鞋,天空碧蓝。这条路通往我的家。

    此时班上最阴郁的那个学生拦住了我的路,却又小心翼翼地领着我走,似乎是用了第三只眼仔细打量着走在后面的我,时刻调整步伐,不让我越过他。忽然他转过头来,从口袋里拿出一本册子,一支笔,望着我,记录我无声的自首,“说吧,您……您还是快说吧”他不耐烦地催促着我,我不敢看他,只好望向那个脱鞋的人,壮着胆确认他的问题,“你怎了?你想问些什么?”,他没有说话。

    他望着他,不看我。过了很久,他谦虚地问我鱼的发光强度,我更加骇起来,只好向那人借了他鞋子里的刀,杀了这学生。

\newpage

\textbf{\#}


\mbox


    \rotatebox {85}{\(f(x) \geq \mathrm{e}^x-3x\)}
    \rotatebox {85}{\(g(x) \leq 1-\sin x\)}
    \\
    \\
    \\
    \\
    东奔西走只为摆脱自我本身。\\
    \textcolor{blue}
    {世界呵!你又走过了一天,\\
    疲惫的我,\\
    流吧\\
    你这淡淡的血、淡淡的泪。}

\newpage
\textbf{\#}


\mbox


    一切有价值的都烟消云散了,而今价值已经没有价值了。但人应当积极向上,哪怕上面并没有美好,哪怕我的积极仅仅只是徒劳,因之囚人在冰冷的室里就算找到一丁点火光也会高兴不已,然而将这在室里的生活寄托于这缥缈的火,恐怕正如天上的繁星般不可靠,人的号叫在这火败灭后化作悲风,托着更憔悴的冬风拂过其他东西的面庞。

    涉江采芙蓉的人已死去,而芙蓉却再生,这大地上的人已换了一轮又一轮,但苦难却不朽。

    一只航船除了船员、螺桨以及各种我不可知的器件以外,还应配有一个彼岸,海上那么多的陆地必须要有一个属于这船,当然这船可以找到那作为彼岸的此岸来停靠,然而这样就并非航船了。\(^*\)


\mbox


    {\normalsize * 编纂时我认为此段应该删去。}

\newpage
\textbf{关于自杀}


\mbox


    终于我还是从众多雅称之外选择了这个朴素而惊悚的本名——自杀。当然并非我想实施一次自杀,我还有很多事未竟,至少当下我不会自杀……也许在很久很久以后?不,我说我应当永远不会做出此事。本文中,我斗胆想对这个主题作些文字,此文不叫《自杀论》,我无法认为我的水准能达到“论”的层次,照我写下这行字时的设想,本文应当不会讨论任何在分析层面的观点,纯粹而言这只是我发发牢骚。

    为什么人会自杀呢?没有人生下来便想死的,从未听闻婴儿自杀的。我不作分析,这纯粹是由个人内心与精神与集体、社会、世界不相容导致的。他们——自杀的人——或许想过这世界在他们心中的样子,然而他们或许——一定?——明白这是不存在的、虚渺的、不可能的。他们却少了一种愚蠢:顺应潮流。既在这社会里变做你应当做的事,毋需构想新东西,那是不必要甚至乎邪恶的。然而他们正是没有这种愚蠢——是的,虽然在当前的社会语境下,你称此为智慧——才没入了自杀的归宿。这是悲哀吧?不是,他们虽然没有这种愚蠢,却有另一种愚蠢——执著,毕竟,“人生来平等”,是不?这种公平的另一种“愚蠢”使他们简直毫无顾忌——不,大多数自杀的人决心并不坚定——又坚定不移地加速自己的人生直至于寂静或伟大的死。

    自杀是不对的。

    自杀是唯一的。

    我不知道如何说了——自杀者无法也不配在这世上存活。正当他们决定了自己的死亡时,世界也给他们判了死刑。他们曾被迫成为世界的一部分,如今他们主动离开。


\mbox


    \hfill 一月十二日\ \ 雨夜

\newpage
\textbf{现代小说的空间叙事}\footnote{\normalsize 有感于语文考试引文《从图像到文学:中国现代小说的空间叙事》}


\mbox


    空间叙事或曰多程叙事,是相对于传统线性叙述而存在的,这类叙述方法的显著特征是在以时间为轴的专义性之外而存在的泛义性、无焦点性。传统小说的脉络犹如精心设计的阿里阿德涅之线,引导着读者向重心滑去,尽管也会有旁支的内容与事件,不过那些被模糊了的“插曲”在主旋律中显得微不足道,这些余光里默默起舞的侧叶通常都会以在某位学者的文章中化作剖析的主题为终局消逝在普通读者眼里,光荣的完成衬托并突出主题的一生任务。而对于现代小说而言,时间轴是被狂暴涌入的景观与现象无情砸毁的遗迹,读者或许能在现代小说中找到时间与顺序的片影,但无疑他们更多时候面对的只是被放大无数倍后的客观,此时作者并不给出一个重心、一个质点而只是提供洪流般的意觉,或说,主题本身就沦为了侧枝旁叶。读者在现代小说中游荡,恰如现代小说在读者头脑里游荡一般,每当读者尝试分析某些情节或人物时,他竟惊奇地叹息,没有,没有安安静静等着他来剖析的人,小说甚至比周围的存在更令人不解。

    空间、空间。纷至沓来的感觉充斥着空间,没有主线,没有意象,没有了所有。

\newpage
\textbf{虫}


\mbox


    七点四十八分,教室里仍有许多飞虫,二十分钟前这群虫便已到教室中,它们飞着,它们被人拍打、吹走、踩碎。人们惧怕,人们厌恶,人们毁灭它们。

    这些虫子的翅膀几乎是透明的,但其中也不乏如反射般的绿光,这种绿光很萎靡,是一种生于腐烂与死亡的令人厌恶的绿。这些烁着绿光的翅膀很容易掉落,然而这些虫凭着这伊卡洛斯的翅翮飞来了,这些虫几乎漫无目的,然而却又似是直奔着白色的东西飞去的,它们落在纸上、墙上、人身上,有时爬着爬着它们那靠不住的翅子便背弃它们离去了,只剩下一米粒大小的褐色躯体在地上蠕动。

    写到此处,这虫几乎已经没有了,之前那些大呼小叫,兴奋不已的人们完成了驱赶虫的使命,教室再一次如昨日般寂静下来,只剩下过去的声响以及一星半点的回忆,这思绪的碎屑之中还夹着不安的翻书声。

    南极洲正经历着日复一日的黑暗,那里的冰海迭起一层波又平复下去,这些寒波在大地的荫蔽下,似乎只是南极洲无穷寂灭中的一染尘,很快,它们便忘却了自我的存在,此后这些浪花只在这纸上求生。
    
\newpage
\textbf{\#}\footnote{\normalsize 正文有许多不完善之处,但因懒惰我尚无修缮之心。}


\mbox


    由 Lagrange 定理知,对性态较好的函数 \(f\) 有如下结果:
    \[
    \forall\ [\alpha,\beta] \subset D,\exists \ \theta \in [\alpha,\beta],\frac{f(\alpha) - f(\beta)}{\alpha - \beta} = f'(\theta),
    \]
    其中 \(D\) 是定义域。
    下面要说明,如果 \(f\) 具备凹凸性,且三阶导 \(f'''\) 在 \([\alpha,\beta]\) 上不变号,则对 \(\theta\) 有如下估计:(以下假定 \(f''>0\))
    \begin{itemize}
        \item \(f'''=0\),则 \(\displaystyle \theta = \frac{\alpha - \beta }{2}\)
        \item \(f''' > 0\),则 \(\displaystyle\theta > \frac{\alpha + \beta}{2}\),对于 \(f''' < 0\) 的情况有类似结果。
    \end{itemize}

    证明中只用到一个容易理解的不等式 —— Hadamard 不等式\footnote{\normalsize 原稿上有图示并说明,但转成 \LaTeX 时我尚未明白 Tikz 宏包如何使用,因而姑删之。}:
    对上述 \(f\),有
    \[
    f(\frac{\alpha + \beta}{2}) \leq \frac{1}{\beta - \alpha} \int^\beta_\alpha f(x)\  {\rm d}x\leq \frac{f(\alpha)+f(\beta)}{2},
    \]\marginpar[Hadamard 是法国人,法语 “H” 不发音,故应读如 “阿达玛”]
    有这个工具后可以开始证明。
    \[
    f'(\frac{\alpha + \beta}{2}) \leq \frac{1}{\beta - \alpha}\int^\beta_\alpha f'(x)\ {\rm d}x = \frac{f(\alpha) - f(\beta)}{\alpha - \beta} = f'(\theta)
    \]
    即 
    \(\displaystyle f'(\theta) \geq f'(\frac{\alpha + \beta}{2})\),注意到 \(f\) 下凸,即 \(f''>0\),\(f'\) 单增,因此 \(\displaystyle\theta > \frac{\alpha + \beta}{2}\),证毕。

    其他情况可作如是讨论。
    另外,当 \(f''' = 0\) 时,\(f\) 为二次函数 \(ax^2 + bx + c\),此时可用初等的计算而毋需用积分证明 (i)。

    
\mbox


    \textbf{\large 小跋}

    {\large 这篇文章是为我一位同学而作,他是化学竞赛金牌\footnote{\normalsize 凭这个信息在我所居住的省份可以锁定到个人了,如果有人在看这篇文章,请允许我向你提出不要打扰他的请求。},当时有一个数学题引起了他的兴趣,前来问我,我在做此题时也感觉到必然有背景,便答应下来。然而我没多少时间思考,一连拖了三周,在某个星期天于宿舍自习室看谢惠明老师的书时迸发出了 “灵感”(这样基本的问题,大家看到了我竟然摆出这两个字大概会笑的吧),遂疾书,呈递之,然而所应不过唯唯耳。我当时还挺伤心的,也夹杂着一些愤怒,并非因他浪费我的时间,而是怪他不理解其中的背景,只关注应试,至少在数学上他的表现一贯如此,关于此人,还有一些可说的,然而我还是决意谨怀揣着敬畏之心就此缄口了。}

\newpage
\textbf{\#}


\mbox


    你看这行字,你读下去,发现她说,“你好,这的确是一段元文字,元故事,我现在意识到我在元故事里,我的话造就出一个元元故事,然后呢?然后我将不断意识下去,正如你所预料的那般忽然打断这过程,但于现世,其实你并没有预料到。我与你确认,只进行到元故事就够了,再往后的世界只不过是作为元故事的上层,单纯复现现世与元故事之间的关联罢了。我的确不知道在此之后的世界会如何,你,您,您是知晓的,自然,您是一目了然了,您只须往后看去,看到末尾便发现结束了,因为世界,再不然作者的兴味总该是有限的吧。看吧,在那引号前面,那是我终末的遗言。正如其他角色一般,我也与您进行对话,我们打破作者设下的墙,但又在另一堵墙中与你进行伪装的对话,‘伪装成对话的独白’。但我与它们是有差别的,我就只在此处,我不在他处,您能理解吗,我要走了,我无心再看这段话了,这种呓语,我离开。”

\newpage
\textbf{仿佛来自虚空}\footnote{\normalsize 这是一模之后我读皮扎尼克《夜的命名术》有感而作,此书对我影响很大。}


\mbox


    寂灭与生长
    
    收割与播种
    
    比起晴空我更爱虚空
    
    是谁摘下了君王与臣仆的面具
    
    让我直面这迷蒙与无穷?
    
    她倾诉着没有没有,
    
    窗外只剩下怜悯众人的空洞

\newpage
\textbf{书诔}\footnote{\normalsize 此文何颜称 “诔”?聊作一笑耳。}


\mbox


    三月二十日书还毕。

    三月二十一日至六月九日晚六时十五分,灭书。

    别了,书。

    {\hfill 三月二十一日谨识}


\mbox


    {\normalsize \textbf{小跋}}

    {\normalsize 当然,我也没有就此与书诀别,二模后我依然很烂,决意干脆读点东西,出人意料,我二点五模成绩倒是很好。}

\newpage
\textbf{一样对黑夜着迷的我们}


\mbox


    山上繁星点点,

    下面的城市依旧从容。

    夜风四散,

    吹进她的窗中……


    \mbox

    
    屋顶下留下了夕阳的温柔,

    酒瓶中留不下半点哀愁。

    我们这样相拥,

    这样在无言之歌里啜泣。

    夜的彼岸 ——

    大概是你的梦吧。


    \mbox

    
    小小的夜幕掩护着她心中的冷清,

    辽阔的都市便随着不尽的夜

    流作一条通向起点的河,

    我们在渐岸旁欢笑,

    水弯弯流向四方。

\newpage
\textbf{\#}

\mbox


    他们站在工地上。前面是一片荒地, 他感受到了风,她没有。


    他说,今天风很大、很凉。她说,不是的。


    他看着工地,就好像工地真的存在一样。


    我看着她,正如我被风看着。我决定杀了他。


    忽然,他消失在风中,同时,她消失在我的回忆中。\(^*\)


    {\normalsize * 以上于在宿舍隔离期间,受 M.D 小说文风影响及她一本我从未读过的作品《工地》之启发而作。}

\newpage
\textbf{梵-我絶對不二論}\footnote{\normalsize 本文作于某节语文作文课上,我素来不喜欢语文作文,因为我想认真讨论一些问题的时候,语文老师告诉我这里没有问题,考试并不需要问题。我而今承认这只是愚钝,在此劝告诸位(可能并不存在的)中学生读者:如果你有很多思想,不要付诸作文,此无异于付诸东流\ ——\ 认真你就输了。

本文参考巫白慧的《圣教论》。}


\mbox


    \begin{verse}

    外慧處外曰周遍
    
    内慧居内曰炎光
    
    深慧住深曰持静
    
    唯一住此三分中
    
    \end{verse}

    梵是宇宙人,我即是此唯一真梵,梵是無相變,唯一於此大我中。

    外慧映於質、色、動、業等等物體質料層面,對應於三相之醒態。内慧映於精神層面,是夢態。深慧是極深無意識的暫臨梵熟睡態。
    
    唯一態即是梵,梵即是我,我即是梵。原人的右眼生出無相梵,原人的左眼生出有相梵。梵我合一,内外無異之梵日後無相梵與有相梵歸於至一之我,之大我,之唯一。不死,不滅,無活、無生死、不二、無質無限、無異、無同、無樂無悲無此無彼、無一切。
    
    進入唯一真梵態,則至静無他,宇宙寂滅,唯梵獨存。

    唵。

\newpage
\textbf{\#}


\mbox


    一旦这颗心脏停止了、停下了,那么一切就不存在了,愤怒、哀怨、痛楚、喜悦、期盼、目及阳光下树枝沐风而舞的激动、阅读文字时理解的快感、心中对伟大之人的尊敬,对城市里电光雷雨一切的感觉、神经、回路、知觉、理性以及虚无之虚无都将化为乌有,化为真正意义上的不存在,化为自我指涉的虚无。

    人须知道,自我是无法被自我毁灭的,我所做的一切都无法也不可能使我清除于我自己的思想、认知中,我所有的力量只能毁灭他人,毁灭外物。这世界是相对的而非绝对的,有些人认为这世界已经死透了,而另一些人认为人类正生机勃勃,我若是终结自己的生命,则必一定只感到这世界、宇宙、尘埃、你 —— 这一切在我眼中无可挽回地悲哀地离开,而不会认为我与世界没了关联的根源是我自身消失了。这与被杀不同,这是宣告一切的死亡而非自我的溘然离去,这是世界的讣告。


\mbox


    {\hfill 二月十七日读《蒙马特遗书》有感而作}

\newpage
\textbf{\#}


\mbox


    北方、北方
    
    那儿有我的坟
    
    那儿是我的故乡
    
    你的雨,你的夜
    
    我在雪中流涕


\mbox


    北方没有故事,没有时间
    
    那儿只有过去的血与泪
    
    只有我遗留的诗篇
    
    惨淡的灯,渺小的我
    
    我躺在冰湖里


\mbox


    北方的寒波抹煞了柔浪
    
    悲风束缚鸟儿
    
    我在大地上化作碎片
    
    都是虚空
    
    都是捕风


\mbox


    {\normalsize \textbf{小跋}}

    {\normalsize 我的故乡不在北方,我也从未去过,这是我脑海中的景象,这些意象大部分源于民防乐队的歌,以及互联网上的 Doomer 音乐、视频。二〇二二年九月,我前往一家酒店隔离,我在那儿度过了影响我整个高三的一周,在那里我于互联网上偶然听到了列托夫(\textrussian{Летов})的歌《一切按计划进行》(\textrussian{Всё идёт по плану}),我被歌词及视频中模糊的影像所震撼,那种混乱与无秩序的场景,那种一切都将结束的临近毁灭之氛围,我只是错愕地观看,我只能错愕地旁观,我从来不曾理解什么,我也无法理解什么,那是感觉不到他人存在的七天,我在酒店里孑然一人,阅读着、感受着。}

\newpage
\textbf{简论\#\footnote{\normalsize 这是我一位同学的名字,于此略去。}滤子场在谷爱凌合取-析取范式下对丁真-孙笑川凝聚完备空间中抽象流形类的综合性具象表达对于抽象性反演纤维丛的等价性完全积定理}


\mbox


    谷爱凌合取-析取范式被其不祧之祖惊为天人地应用于现代民族性人类全集 \({\rm Hum}\)的幂集 \(\mathcal{P}({\rm Hum})\)  之阶级子集间双射存在性的强滤子性辅助条件以保证元素在封闭阶级子集流通的交集全备性成立,这一有用的范式很快在\# 滤子场的观点下被发掘出于丁真-孙笑川凝聚完备空间中的大用处,丁-孙空间中抽象流形类在一定程度上构成了谷爱凌极大理想中的局部等价类单射极小全序环上不动点空间的一簇无穷仿射不变性算子基,这些算子基又在具象语义表达下构成\# 一阶逻辑自然推演-自我指涉性转化生成前驱病态反演纤维丛,理所当然地成为诸抽象性研究机构之讨论专题集中学习谷爱凌合取-析取范式的另一剖析工具,因此我们得以在民族性人类全集之特定子集中不同地域性位点等价流形转化的研究中抢占先机。本文旨在介绍抽象性反演纤维丛上的等价性完全积定理及其公理格式塔式结构的极大不变量。
    
    ……\(^*\)

    {\normalsize *下文暂缺。}

\newpage
\textbf{序}\footnote{\normalsize 这是我高中某曾任校长题给当年毕业同学录的序,感人肺腑,录以自勉。另外原文繁体,且多异体,我尚不知如何美观地输出异体字,姑用简体录入。}

    戊寅夏,初中三年级毕业,将刊同学录,请序于余,余虽学问无穷,事功有待,乃进诸生而诏之曰:诸生入校已历三载,朝夕乾惕,晦冥不倦,其于业也可谓勤矣;春不避阴雨,夏不避暑热,冬不避寒冻,其为气也可谓勇矣;流离迁徙,杖履相随,历风雨而弗渝,凛冰霜而独劲,其立志也可谓坚矣。艰难困苦,卒抵于成,不亦休乎。虽然,初中毕业,对于普通科学,略涉岸涯而已,尚未造乎大成也。诸生果未安小就,益求上进焉,由敬业乐群而至于知类通达,从学问思辨而达乎高明悠久,则须毋绥而志,毋暴而气,毋见异而思迁,毋半途而中辍。盖为学之道,譬如逆水行舟,不猛进,斯倒退矣;譬如登峰造极,苟畏难,斯落伍矣,诸生勉乎哉!
    
    诸生行将离校,散处四方,恐不能复相聚于一堂。爰刊学录,以志其里居姓氏,俾克通音问,而联情愫,其用意固可嘉也。异日者,人手一册,于船唇车轼,一披览焉,曰:某也直、某也谅、某也多闻,吾将往而切磋焉;又曰:某师博学多能,可以沦吾智慧也;某师豁达大度,可以拓吾胸襟也;某师规行矩步,可以资吾楷模也,吾将往而就正焉。庶于相长相成之义有合也,诸生勉乎哉!

\newpage
\textbf{人群中的人与被欺骗感}


\mbox


    我感觉如今的日子与初三越来越像,不同的是我更加忧愁,成绩的状况也不如初中前两年般……良好。

    初三的时候——主要是下学期——我有一种很强烈的被欺骗感,我经常枕在枕头上独自流泪,我不明白,我读不懂到底发生了什么,我只知道一件事:没有人能或愿意让我知晓事件的本末、经过。我躺在床上想,我想不明白,遇到了他人,他人也不理解我,我与众人似乎被割裂开来了,我们并非同一种人,我……我长久地无法接受。

    而今,我高三了,坐在这间三年前我就常来的小房间里,面对一张来去无人关心的纸,写下这些无人应答的文字——恰似是这文字主人的生活一般。我甚至觉得好笑,果然,何人撰何语是么。我的日子似乎重返了初三,我也有一种很强烈的被欺骗感,我做出过比当时更激进的行为,我有过更绝望更残忍的念头,有些日子我发了疯,我几乎不认识自己了。好在,好在我而今再次平复下心情,同我最熟习的朋友叙旧——孤独。初三的时候,我也是独自一人,不,不,那时我尚有同桌,是的,左边是胡君安,右边是关文韬。一天晚上,我把胡君安的书拿来看了,整整一个晚上,我看了这么久,我当时坐在第一排的,我不知道大家坐在哪里,我只记得了他们两个,以及我回白云广雅前与关文韬的依依不舍,我那时似乎感到了在这里,在广雅高中的第一个朋友,然而这第一个朋友如今早已与我无甚关联,我们几乎再也没说过话,如今他是广州市的第七名,而我是广州市的第一千六百名。

    这段时间,我心中再次产生了一种被欺骗感,我如今不是被某个人欺骗了,我如今是被一切人欺骗了。加入生活欺骗了你,不要灰心,不要失望,幸福的日子很快就会来到。好,我不灰心,我等着幸福的日子。

    我发觉原来我一直是人群中的人,然而不管了,无所谓了,我还有孤独。

\newpage
\textbf{\#}


\mbox


    弹尽粮绝的一日胜似一日

    眢枯的双目噙着眼泪

    鲫鱼啊鲫鱼

    镜中的人当真是我自己?

    镜外的陌生人大笑

    我早已不认识你

    鲫鱼只是前往深处

    淡水还是海水

    无论哪种

    鲫鱼只是前往深处\(^*\)


\mbox


    {\normalsize *四月二十二日做深圳生物题时读到,“……鲫鱼可在缺氧、寒冷的环境中生活数日”}


\newpage
\textbf{\#}\footnote{\normalsize 这是一段《李凭箜篌引》的评论,原意按原文竖排,奈何技术实在太差,看了网上的指导还没学会。}


\mbox


    董懋策《評》:説得古古怪怪。分明説李憑是月宫霓裳之樂,却説得奇怪。

\newpage
\textbf{\#}


\mbox


    晚上六点半,一名高中生走进图书馆,此时读者正期待着我说下去,读者对这名学生即将实施的行为、他的过去、他的未来感到好奇并诧异于作者异乎寻常地步入文本本身与他们对话、预测他们的思绪或者完全胡说一通的意义,我要说:没有。我并不知道,我并不会告诉读者一个故事,这里没有故事,读者唯一能看到的就是作者本人,读者唯一能想象到的是作者正在写作。这段文字正在描述他本身——还是它本身?作者不使用幻象、手法、修辞叙述一件故事。读者认为这是模仿汉德克的拙劣作品,读者对这段文字不满。读者离开。

\newpage
\textbf{3 月 12 日,有感于现代艺术作品《玫瑰》}


    在我面前巍然矗立的是十二盏电灯,这些发光的东西等分了一个光环,不错,十二盏电灯映照着粉红色的墙,我身旁一个人也没有,我独身一人站在这电灯环前面,没有开关,我不知道他们为谁而亮,还要亮多久,这是一项艺术,我低下身子抚摩光环,抚摩着炽热明亮的墙,然而我伸手触摸这无形的天国信使时,它猝然消失,我听闻一阵轰鸣的寂静,随后什么也没有发生,我卑怯地缩回手,被电光荧荧而神圣的亮光恩准过缩回来的手,一刹那,或是一天,我欣喜地立即在天国大门前跪下,祈祷着他们的审判与裁决,在这期间我一动不动,我虔诚而持久地等待他们对罪人的控诉与教导,最终用权力把我抹煞于这世上,我不敢抬头,至尊的权威正在居高临下地审视着我,这是凡人最终的荣光与梦想,我默然抽泣,流下感激的泪,他们给予我生命,甚至允诺我死亡,我始终小心翼翼地命令泪水只在我脸庞流淌,我决不允许它们流下跌落以破坏法庭的尊严与肃穆,马上就要成功了,他们对我的命运之去已然决定,我享受着这永恒的寂静,我沐浴在万世永垂的辉耀之中,终于我不在制止泪水,我赏赐它们肆意奔流,幸福与希望在此刻一齐来到,我终于步入的荣誉的殿堂,我纵身一跃和他们合为一体。

\newpage
\textbf{\#}


\mbox


\begin{multicols}{2}

    天空下
    
    躺着我
    
    海洋中
    
    埋葬着我的骸骨
    
    我在何处?


\mbox


    在这单人的囚室里

    种下一颗种子

    杀了他!杀了谁?

    我说,

    这是永恒的

    春天



\mbox


    冬天来了

    窗外的居民楼

    屹立在寒风中不倒

    窗里的居民们

    沉沉睡去……

    北风嘲笑太阳

    太阳逃向远方

    
\end{multicols}

\newpage
\textbf{夜晚}


\mbox


    夜色并非如诸位所幻想的那般倏忽降临,反而是一点一点地笼罩人的内心、麻痹我的灵魂。有时我长久地望着窗外,期待能抓住夜晚降临的那一瞬间,但这么久以来却没有成效,每次我都陷入沉思中,把将来设想为过去,陶醉在紫黑色天空所营造的梦幻中,我似乎从未从这沉寂而甜蜜的腐烂中发觉我早已失去生活的焦点,但夜晚怎么会怜悯我哪怕一分,夜晚拖着轻盈的步伐朝远方走去,只剩下永恒而炎热的白昼从夜晚的施舍中逃出人的睡梦,苟延残喘地提醒我一切都已支离破碎,生活不再是以往的绝望与悲哀,而是没有选择且幸福的满目凄凉。

    被夜晚抛弃后,我常常在白昼里无所事事,我痛恨这空洞的荒原,这喧嚣的闹市,可是我到底有一个精神支柱,我期待着光辉的泯灭,时间、时间、我没有时间……

    爆发出来的情感,让我这个人群中的人几乎就要消失在世上,因此我成了夜晚的常客,我迫切地渴求夜晚把我裹住,让夜晚的沉寂与冷漠使白昼地炽热吞噬而尽。我曾经把眼泪付诸夜晚,付诸没有回音的渊崖,这狂放的暴君几乎诱使我一跃而下,享受永远的呼啸,也许,今夜我便同这无言的夜色融为一体。

\newpage
\textbf{男孩}


\mbox


    我走在一条大抵说得上是路的水洼道上,这条路通向我要去的地方,我走在上面,向那我须到达的地方行进,其实我并未察觉到所有这些东西,我好似行走在任意一处地方,只有抵达了目的地我才发觉,才醒悟我身处何方,此前在路程之中我的感官是不存在的,我甚至感受不到我本身的存在。是的,我走着,我没有吼叫、哭泣、悲哀,我仅仅是在没有质料的路上走着,水洼道上有水,水洼道是一条通向某处的通道,我只知道这些。我没有与任何人相遇、谈话、争执,我旁边没有人,我不孤独,我什么也没看见,我走着,我前进着。

    在这路上走来一个男孩,他手里拿着什么东西,发出很重的声响,似乎不休止、无止境地响着,好像受了什么刺激一般在咆哮,在倾诉。我听不懂,然而我被唤醒了,我这旅人在水洼道上被重击而惊醒了,霎时间我的感觉如同夏日里狂暴交媾的蓟草,它们在无人的院子里趁主人不备肆意进攻、防备、繁殖,它们在阳光下不断伸长,这些最痛苦的从刺激中作乐的杂草在空间中延申开来,形成一段另一段的墙,它们朝主人的肉体进攻,把主人的尸骸扯成碎片后消化殆尽,把枝叶的细丝伸进主人残余的血管和腐烂的骨髓中吸食过往的一切,最终把大脑的一道道沟回填满再撕裂、踩碎,重建另一个植物的躯干并再次以不可逃避的威严命令它不攻自破,销为尘烬。

    男孩戒备地抬起头,以惊惧而警惕的目光扫了我一眼,此后我们匆匆离去,像不曾见过一般。

\newpage
\textbf{\#}


\mbox


    \textcolor{blue}{\textsf{,fuck 'all}} {\normalsize \textsf{\emph{noun.}}} \textsf{[U] (BrE, tatoo, slang)} (冒犯语)他妈的一点没有,他妈的根本没有 \emph{You've done fuck all today.} 你他妈的今天啥也没干。
    \\


    \mbox

    
    \textcolor{blue}{\textsf{fuck\!\!\!\! · \!\!\!\! ing}} /ˈfʌkɪŋ/ {\normalsize \textsf{\emph{adj.,adv. (taboo, slang)}}} a swear word that many people find offensive that is used to emphasis a comment or an angry statement. (加强语气)该死的,他妈的。 \emph{I'm fucking sick of this fucking rain!} 这该死的雨真他妈的让我心烦!{$\diamondsuit$} \emph{He's a fucking good player.} 他是个他妈的优秀球员。
    \\


    \mbox

    
    \textcolor{blue}{\textsf{'fucking well}} (\emph{\textsf{especially BrE}}) used to emphasize an angry statement or an order (强调愤怒或命令)\ \ \emph{You're fucking well coming whether you want to or not.} 不管你想不想来,你他妈的也要给我过来。

\newpage
\textbf{\#}


\mbox


    世界始于毁灭,无终结,即世界一直处在毁灭中,其中无真正意义上的“创造”。一切事物皆生于无序的混乱、混沌,没有尽头,此地即是终点,此时即是起点。物体内里存在着存在着的灵素,诸灵素是毁灭的起因,也是毁灭的产物,灵素有无限多,在一个灵素湮没之一瞬,另一个灵素将与前灵素融合诞生出另一种毁灭,一种毁灭是一种存在、时空、宇宙。毁灭是宇宙本身,这背后不存在任何人、物、神祗操纵着灵素的合与散。毁灭不遵循任何法则,是纯粹混沌的无序,毁灭即是创造之本源,生即是死的归宿,这是没有起点没有终点的毁灭。

    我即是一种毁灭,我并不认识其他毁灭,我并不领悟诸灵素,我不承认一切灵素。我并不存在,灵素并不构成什么,灵素只永远存在于毁灭中,一切皆无神奇,万物皆是毁灭的不同表达。没有此刻、此时,没有虚无,没有存在。

    这便是世界,这便是欣欣向荣的世界,这便是寂灭与生长。

    在我扫过琴弦之时,无量世界于我指尖诞生,无量宇宙在弦间毁灭。

    灵素的毁灭并非原质的赞歌,而是诸人所谓原质之挽歌,这毁灭本身就是赞歌。


\mbox


    \hfill{一月十九日阅《薄伽梵歌》后闻西塔尔琴音作}

\newpage
\textbf{\#}


\mbox


    风拂过大地

    未曾留下丁点儿痕迹

    我来到人间

    作下这寂静的诗篇

    风为天地之使

    我是生死之司

    她掠过我的面庞

    我嗅到风的芬芳

    总有一天

    风会回到高高的蓝天上

    那时

    我便吻着生命的海浪

    划向远方\(^*\)


\mbox


    {\normalsize *某日于澡间中得灵感(此物审存乎?未知)而作。}

\newpage
\textbf{\#}


\mbox


    人来到这世上一片黑暗

    人蒙住双眼

    把这称作无限的光明


\mbox


    世界的本来面貌

    不在此,也不在彼

    而在人用来蒙住双眼的

    手里


\mbox


    只有盲人、瞽目者才终日见到黑暗

    人给予他们怜悯的泪

    黑夜嗫嚅:

    血!这可是血啊……


\mbox


    盲人终究倒在这眼泪里

    \hfill 二月七日

\newpage
\textbf{\#}


\mbox


    人应当活着,即使面临困难也不能轻易放弃……是么。有时我只是感到无聊、无力,我不想和他人交流,实际上人与人之间很难交流,日常的谈话只是照例行事以作证在种种社会机构中我这个部件任有可利用性,至于你心中、你脑海中的思绪,你的苦恼以及其他种种反正从生下来的那一刻也不会有人关心……有时我真想跑到无人的山顶上静静地在心里对自己说话,想想自己的一生,不用迫使与他人照例行事,世界即是自己,自己即是世界。我还是喜欢自然甚于人,自然简单、平凡、包含万物。好吧,生活还要继续,并不停下来等我一人,不想在这滚滚车轮上被摔下、碾死,就必须奔跑、奔跑,直至死亡。\(^*\)


\mbox


    {\normalsize *以今天的眼光\footnote{\normalsize 2023 年 8 月 19 日,心情愉快。} 来看,略显悲观(并不蕴含“这不真实”的话外之音)。}


\newpage
\textbf{跋}


\mbox


    这是名副其实的跋,上面这些文章,这些我写出来的东西都是很久很久以前的思绪了,这些纸上的舞者陪伴了我的高三,我的灵魂聚积于此,我的痛楚长眠于斯。

    我第一次写下这些琐碎的话语时候,也正是我第一次阅读玛格丽特·杜拉斯的时候,她的《情人》影响着我,改造着我,“我爱你备受摧残的面庞”、“太晚了,太晚了,在我这一生中,这未免来的太早,也过于匆匆。才十八岁,就已经是太迟了。”这些话语流经我干涸的心灵,多么美妙的文字啊!从那时起,我开始留心生活,我曾经敏感的心又被唤醒,我变得热烈,我开始痛苦,高三无意义的考试、病态竞争却又沾沾自喜的大家,我以最热烈的目光迎接这青春的芬芳,然而我只能看到一篇荒凉、万物萧条,这里没有任何思想的附着点,我们无依无靠地停泊在这永恒的荒原上。

    高中三年我什么也没有学到,而今回想从前,只见一堆破碎的笑话在风中抓住记忆的最后回闪,奋力唱着闻所未闻的歌谣,它们试图证明意义之所在,最终只证明了意义不再。那些导数,那些费尽心思的尔虞我诈,那些讨好一切的趋炎附势,那些对着\emph{毫无意义}的题目俯首称臣的我消逝了,未来是这样狠狠地把这些残存的絮语扫空、毁灭,我顾盼这一年,只能看到起点,中间的事情我却一无所知。

    那些诅咒、委屈、愤恨、不满,这些也一样于无人处消逝了。

    寒假的逃脱,我像是哼着“我死后哪怕洪水滔天”的皇帝一样离开了学校,那些我敬爱的人们、我蔑视的人们、我所珍视的人们,我一一抛弃了,我所知道的,我所能感受到的只是空荡荡的城市,一切都没有变化,一切却都变化了。在家里,我与题目诀别,在网上伸出我的神经探到世界的一角,我只发觉——原来生活是这样的……寒假,那个冬天,那段历史,那是幸福的时光,我牢记于心。

    后来无非与以往同类而已。

    如今这一切都被洪水淹没,我的思绪也如布满大地的尘土——它们离我而去,它们停留在原来的地方。一切都如常,陪伴我的文字,感谢你们,你们陪我度过了这首我不愿回想、只能\emph{感受}的青春之歌,没有你们这些充斥于我脑海中的幽魂,没有你们这群令我一读再读的文章,我不会坚持到现在,我的生命已经与你们息息相关,这片天空——这片我看到时会想起从前、想起梦想的天空——由你们存在而更加清澈。空中漂泊不定的,那是我的青春。


\end{document}
